\documentclass{easychair}

\usepackage{lipsum}
\usepackage{amsfonts}
\usepackage[utf8]{inputenc}

% \title{Higher Order Constraint Logic Programming with Applications to Interactive Theorem Proving}
\title{The pragmatic construction of ELPI}
\author{Ferruccio Guidi \and Claudio Sacerdoti Coen \and Enrico Tassi}
\institute{
  Department of Computer Science and Engineering, University of Bologna \email{ferruccio.guidi@unibo.it} \and
  Department of Computer Scienc and Engineeringe, University of Bologna \email{claudio.sacerdoticoen@unibo.it} \and
  Inria Sophia-Antipolis, \email{Enrico.Tassi@inria.fr}}

\authorrunning{F. Guidi, C. Sacerdoti Coen, E. Tassi}
\titlerunning{whatever}

\begin{document}
\maketitle

\begin{abstract}
Higher Order Logic Programming and its main incarnation, which is $\lambda$Prolog, were introduced by Nadathur and Miller in the 80s. Like logical frameworks, it allows to write type-checkers extremely naturally, just encoding the derivation rules in an almost verbatim way. In particular, thanks to a shallow encoding of binders, the implementor is relieved from dealing with binders, capture avoiding substitution, alpha-conversion and the like. By delegating them to the language, it can focus on the logic of the program he is implementing and he can hope in future optimizations of the compiler to improve the efficiency of code. Thanks to the Curry-Howard isomorphism, proof-checkers --- e.g. for dependently typed languages --- can also be encoded in the same way.

To implement an interactive theorem prover following the Curry-Howard correspondence, the programmer basically has to generalize the type-checker by allowing incomplete terms whose holes are represented by existentially quantified metavariables. Adopting a deep encoding, the user is forced to implement non-capture-avoiding instantiation, explicit substitutions and higher order unification under a mixed prefix. Therefore, we would prefer a shallow encoding of metavariables, where the existentially quantified metavariables of the logic programming language directly represent metavariables of the encoded typed calculus, and the previously mentioned operations can be largely inherited from the metalevel.

Such deep encodings do not work for $\lambda$Prolog and, more generally, for all higher order logic programming languages and logical frameworks. Indeed, the standard operational semantics of $\lambda$Prolog is generative, i.e. when a predicate like well-typedness inspects a flexible term, all possible instantiations of the latter are tried blindly. On the contrary, in an interactive theorem prover a predicate over a flexible term is supposed to be delayed and turned into a constraint to be solved later. When constraints accumulate, they can be simplified and propagated exactly in the spirit of constraint programming.

In this paper we present a first proposal for an higher order constraint logic
programming language obtained as an extension of $\lambda$Prolog. We also present
applications to the implementation of an elaborator for the Calculus of
Inductive Constructions, i.e. the core of a modern interactive theorem prover
in the style of Coq, Agda or Matita.
\end{abstract}

%%%%%%%%%%%%%%%%%%%%%%%%%%%%%%%%%%%%%%%%%%%%%%%%%%%%%%%%%%%%%%%%%%%%%%%%
\section{Introduction}

\paragraph{A pragmatic reconstruction of $\lambda$Prolog.}

In~\cite{jlp98} Belleannée et. alt. propose a pragmatic reconstruction
of $\lambda$Prolog~\cite{lambdap1,lambdap2,lambdap3}, the Higher Order
Logic Programming (HOLP) language introduced by Dale Miller and
Gopalan Nadathur in the '80s.
Their conclusion is that $\lambda$Prolog can be characterized as the
minimal extension of Prolog that allows to program by structural
induction on $\lambda$-terms. According to their reconstruction, in
order to achieve that goal, Prolog needs to first be augmented with
$\lambda$-abstractions in the term syntax; then types are added to
drive full higher-order unification; then universal quantification in
goals and $\eta$-equivalence are required to express relations between
$\lambda$-abstractions and their bodies; and finally implication in
goals is needed to allow for structural induction definitions of
predicates.

By means of $\lambda$-abstractions in terms, $\lambda$Prolog can
easily encode all kind of binders without the need to take care of
binding representation, $\alpha$-conversion, renaming and
instantiation. Structural induction over syntax with binders is also
made trivial by combining universal quantification and implication
following the very same pattern used in Logical Frameworks like LF
(also called $\lambda$P). Indeed, LF, endowed with an HOLP semantics
like in ???(ELF? Beluga? MMT?) is just a sub-language of
$\lambda$Prolog.

The ``hello world'' example of $\lambda$Prolog is therefore the
following two lines program to compute the simple type of a
$\lambda$-expression:

\begin{verbatim}
kind term type.
type app term -> term -> term.
type lam term -> (term -> term) -> term.

type arr typ -> typ -> typ.

type of term -> typ -> o.

of (app M N) B :- of M (arr A B), of N A.
of (lam A F) (arr A B) :- pi x\ of x A => of (F x) B.
\end{verbatim}

\paragraph{$\lambda$Prolog for proof-checking.}

According to the Curry-Howard isomorphism, the program above can also
be interpreted as a proof-checker for minimal propositional logic. By
escalating the encoded $\lambda$-calculus to more complex terms and
types, it is possible to obtain a proof-checker for a much richer
logic, like the Calculus of Inductive Constructions that, up to some
variations, is the common logic shared by the interactive theorem
provers (ITPs) Coq~\cite{}, Lean~\cite{}, Matita~\cite{} and
Agda~\cite{}. For example, in~\cite{us} we implemented in
$\lambda$Prolog a type-checker for Landau's Grundlagen XXXXXXXX.

Indeed, all the ITPs mentioned above are implemented following
basically the same architecture. At the core of the system there is
the \emph{kernel}, that is the trusted code base (together with the
compiler and run-time of the programming language the system is
written on). The kernel just implements the type-checker together with
all the judgements required for type-checking, namely: well formation
of contexts and environments, substitution, reduction, convertibility.
The last three judgements are necessary because the type system has
dependent types and therefore types need to be compared up to
computation.

Weak head normal form is straightforward, since one can use the
$\beta$ reduction of $\lambda$Prolog to implement the one of the
object language:

\begin{verbatim}
type whd term -> term -> prop.

whd (app M N) R :- whd M (lam _ F), whd (F N) R.
whd (lam T F) (lam T F).
\end{verbatim}

The conversion test is a bit more delicate, since it is a performance
critical piece of code (see section~\ref{sec:kernel}).  Here we code
the usual strategy of normalizing terms only if really necessary (what
systems typically do since Automath).

\begin{verbatim}
type conv term -> term -> prop.

conv X X.
conv (app M1 N1) (app M2 N2) :- conv M1 M2, conv N1 N2.
conv (lam _ F1) (lam _ F2) :- pi x\ conv (F1 x) (F2 x).
conv T (app M N) :- whd M (lam _ F), conv T (F N).
conv (app M N) T :- whd M (lam _ F), conv (F N) T.
\end{verbatim}

To obtain a checker for an object language with dependent types and
a predicative hierarchy of sorts we need to add a new term constructor
and refine the type of \verb+arr+ as follows:

\begin{verbatim}
type arr term -> (term -> term) -> term.
whd (arr T F) (arr T F).

type sort int -> term.
whd (sort I) (sort I).
\end{verbatim}

Both \verb+arr+ and \verb+sort+ are rigid (in head normal form), hence the need to extend the \verb+whd+ predicate.  The conversion test, in presence of
a universe hierarchy needs to take into account variance and cumulativity.
With little effort we define \verb+sub+, the entry point for the conversion 
test, taking care of comparing sorts appropriately.

\begin{verbatim}
type sub term -> term -> prop.
type sub-whd term -> term -> prop.

sub A B :- whd A A1, whd B B1, sub-whd A1 B1, !.
sub-whd A B :- conv A B.
sub-whd (sort I) (sort J) :- I < J.
sub-whd (arr A1 F1) (arr A2 F2) :- conv A1 A2, pi x\ sub (F1 x) (F2 x).
\end{verbatim}

Finally, the typing rules for all that.  Of course we need to mix
terms and types, hence relax the type of \verb+of+.

\begin{verbatim}
type of term -> term -> o.

of (app M N) BN :-
  of M TM, sub TM (arr A1 Bx), of N A2, sub A2 A1, BN = Bx N.
of (lam A F) (arr A B) :-
  pi x\ (of x A, whd x x) => of (F x) (B x).
of (sort I) (sort J) :- succ I J.
of (arr A Bx) (sort K) :-
  of A TA, (pi x\ (of x A, whd x x) => of (Bx x) TB),
  sub TA (sort I), sub TB (sort J), max I J K.
\end{verbatim}

where \verb+succ+ and \verb+max+ are the predicates governing sorts in
PTS.  We now have the predicative, universal fragment of Luo's
ECC~\cite{luo}, wow!

Note that the rules are syntax directed, i.e. with explicit calls to
\verb+sub+.  The on-paper presentation is more elegant and compact but 
uses in a fundamental way non determinism (hence results in too much
backtracking) by simply adding:

\begin{verbatim}
of A B :- sub B C, of A C.
\end{verbatim}


\paragraph{From proof-checking to interactive proving.}

The kernel is ultimately responsible for guaranteeing that a proof
built using an ITP is fully sound. However, in practice the user never
interacts with the kernel and the remaining parts of the system do not
depend on the behaviour of the kernel. Where the real intelligence of
the system lies is instead the second layer, called \emph{elaborator}
or \emph{refiner}~\cite{??,??,??}.


In this paper we investigate if $\lambda$Prolog can do that too, if not
what shall be added.  We start by detailing what an elaborator is and
why the state of the art is not satisfactory.  We then present the
issues
we faced when coding an elaborator in $\lambda$Prolog and how similar issues are
dealt with in first order Prolog.  We then present ELPI, our extension to
$\lambda$Prolog.  Finally we present a kernel and a prototype refiner for CIC
written in ELPI.  We then conclude.

\section{The elaborator component of today's ITPs}

An elaborator takes in input a \emph{partial term} and optionally a
type, and returns the closest term similar to the input one such that
the term has the expected type. Both the input and output terms are
partial in the sense that subterms can be omitted and replaced with
named holes to be later filled in or, in logic programming
terminology, with \emph{existentially quantified metavariables}. For
example, the partial term $\lambda x: T. f~(X~x)~Y$ where $T,X,Y$ are
quantified outside the term represents the $\lambda$-abstraction over
$x$ of a type yet to be determined of the application of $f$ to two
arguments, both to be yet determined and such that $x$ can happear
free only in the first. Elaborating the term versus the expected type
$\mathbb{N} \to \mathbb{B}$ will instantiate $T$ with $\mathbb{N}$ and
verify that $f$ is a binary function returning a boolean.

The importance of the elaborator is twofold. On the first hand, it
allows to interpret the terms that are input by the user, usually by
means of a user-friendly syntax where information can be omitted,
mathematical notation is used and abused, subtyping is assumed even if
elements of the first type can only be coerced to elements of the
second by inserting a function call in the elaborated term. A better
elaborator therefore gives to the user the feeling of a more
intelligent and user friendly system. On the other hand, via
Curry-Howard, a partial term is a partial proof and an ITP is all
about instantiating holes in partial proofs with new partial terms to
advance in the proof. The elaborator is thus the mechanism that takes
a partial sub-proof and makes it fit in the global one to progress on
a particular proof obligation. In other words, all tactics of the ITP
ultimately produce partial proof terms that are elaborated. The more
advanced is the elaborator, the simpler the code implementing tactics
can be.

\subsection{Implementing an elaborator: state of the art}  %%%%%%

The elaborators of the majority of the provers mentioned above are all
implemented according to the same schema: the syntax of terms is augmented with
explicitly substituted existential variables
and the judgements of the kernel are re-implemented from scratch, generalizing
them to take in account metavariables and elaboration.

In particular one no more works with terms, but always with partial terms and
a metasenv: a data structure assigning to metavariable a typing judgement
and, eventually, an assignment.

All algorithms manipulating terms are extended to open
terms, in particular conversion (implemented by the 
\verb+sub+  and \verb+conv+ predicates in the running example)
becomes narrowing, i.e. higher order unification under a
mixed prefix~\cite{???} in presence of rewriting rules.
Unification requires metavariable instantiation, that is
implemented lazily for efficiency.  Type checking \verb+of+ is generalized to
elaboration, for example by replacing all calls to conversion with calls to
narrowing and by threading around the metasenv.

The approach is sub-optimal in many ways.  We here identify the main problems
in the state of the art.

\begin{description}
\item[programming platform]
Much of this new code has very little to do with the prover or the implemented
logic. In particular code that deals with binders ($\alpha$-conversion,
capture avoiding substitution, renaming) and code that deals with existentially
quantified metavariables (explicit substitution management, name capturing
instantiation).
\item[intricacy of algorithms]
Such code is far from being trivial, since it tackles problems that, like
higher order unification, are only semi-decidable. For efficiency reasons a lot
of incomplete heuristics are implemented to speed up the system and reduce
backtracking. The heuristics, however, are quite ad-hoc and they interact with
one another in unpredictable ways. Because they are hidden in the code, the
whole system becomes unpredictable to the user.
\item[code duplication]
Given such complexity in the elaborator, and the safety requirements of interactive provers, the kernel of the system is kept simple by making it unaware of partial terms.  As a consequence a lot of code is duplicated, and
the elaborator ends up being a very complicated \emph{twin brother of the
kernel} (Huet's terminology).
\item[twins' disagreement]
Worse than that, such twin components need to agree on ground terms.
Typically a proof term is incrementally built by the elaborator:
starting from a meta variable that has the type of
the conjecture the proof commands make progress by instantiating such
meta with partial terms.  Once there are no unresolved meta variables left,
the ground term is checked, again and in its totality, by the kernel.
Bugs in the elaborator are detected by the kernel, letting the user 
\item[extensibility of the elaborator]
Finally, the elaborator is the brain of the system, but it is oblivious of the
pragmatic ways to use the knowledge in the prover library, e.g.  to
automatically fill in gaps~\cite{mathcomponents}, to coerce data from one type
to another~\cite{coercivesubtyping} or to enrich data to resolve mathematical
abuse of notation~\cite{nonuniformunificationhints}. Therefore systems provide
ad-hoc extension points to increase the capabilities of the elaborator.  The
languages to write this code are typically high level, declarative, and
try to hide the intricaces of bound variables, metavariables, etc. to the user.
The global algorithm is therefore split in multiple languages, defying the hope
for static analysis and documentation of the elaborator.
\end{description}

\subsection{The proposed approach: semi shallow embedding}

The motivation of our research is to try to improve over the latter issues.  We
propose to identifying an high level (logic) programming language suitable for
the implementation of elaborators. In particular:
\begin{enumerate}
\item The programming language takes care of the representation of
	\emph{bindings and metavariables} in the spirit of semi
	shallow embedding~\cite{Dunchev:2016:IHH:2966268.2966272},
	solving the \textbf{programming platform} issue.
	It also improves on \textbf{extensibility} by allowing
	both the core implementors and the users to work on the same,
	high level code.
	Finally the \textbf{intricacy of elaboration} is be mitigated.
\item The programming language features a primitive and powerful
	notion of extensibility: programs are organized into clauses,
	and new clauses can be freely added.
	In this way the rules of the kernel are not be re-implemented in the
	elaborator. On the contrary they are extended to cover partial terms,
	solving the \textbf{code duplication} issue.
	Also, the \textbf{twins' disagreement} problem becomes less severe,
	since most code is shared.
	At the same time \textbf{extensibility of the elaborator}
	become less ad-hoc: the user simply declares new clauses.
\item Finally the programming language has a clean semantics, making it
	easy to prove that the extensions to the kernel
	only accept partial terms that are, once completed,
	well-typed according
	to the core set of rules of the kernel.  This completely
	solve the \textbf{twins' disagreement} problem, making it possible to
	merge the kernel and the elaborator.
\end{enumerate}

We envisage such language to be a logic one for two reasons:
first we hope to reuse or at least extend $\lambda$Prolog;
second we observe that another component of each ITP, the one
implementing proof commands, can take real advantage from a 
programming language where backtracking is built in, in
particular to write proof commands performing proof search.

The Higher order Abstract Syntax approach identifies the object
language binders and the meta language ones, obtaining $\alpha$-conversion
and $\beta$-reduction for free.  The semi shallow embedding we propose
\emph{identifies the metavariables of the object language with the 
meta variables of $\lambda$Prolog}.  Indeed such meta variables already
come with automatic instantiation and context management.

At a first sight, the runtime of $\lambda$Prolog seems to already provide
an implementation of the metasenv data structure, and all related operations.
Does this idea work out of the box? Unfortunately not quite.

%%%%%%%%%%%%%%%%%%%%%%%%%%%%%%%%%%%%%%%%%%%%%%%%%%%%%%%%%%%%%%%%%%%%%%%%
\section{$\lambda$Prolog meets partial terms}

We already know from~\cite{jlp98} that $\lambda$Prolog is the minimal
extension of Prolog that allows to implement inductive predicates over
syntax containing binders. Does it work when applied to data that is
meant to containt existentially quantified metavariables too?

Consider the $\lambda$-term \verb+(lam a\ P a)+ that encodes a
partial proof of $\forall A, A \to A$.
If we run the following query, the computation diverges:

\begin{verbatim}
goal> of (lam a\P a) (arr (sort I) a\ arr a _\ a)
\end{verbatim}

\subsection{Generative semantics and constraint declaration} %%%%%%%%%%%

Indeed, \verb+P+ is flexible and the \verb+of (app M N) BN+
rule applies indefinitely.

Indeed the \verb+of+ predicate inherits from Prolog a generative semantics:
when called recursively on a flexible input, it enumerates all
instances trying to find the ones that are well typed. Even when proof
search does not diverge, the behaviour obtained is not the one
expected from an elaborator for an \emph{interactive} prover: the
elaborator is not meant to fill in the term, unless the choice is
obliged. On the contrary, it should leave the meta variable not
instantiated and should \emph{remember} the need for
veryfing if the predicate holds later on, when the metavariable gets
instantiated. In the example above, type-checking \verb+(lam a\P a)+ forces
the system to remember that term of type \verb+(arr a _\ a)+ has to be
provided, that in turn corresponds to the proof obligation
$A : o \vdash A \to A $.

This is not a problem specific of $\lambda$Prolog since Prolog behaves the same.
Nevertheless, all modern Prolog engines provide a \verb+var/1+ built-in to
test/guard predicates against flexible input, provide one/many variants of
\verb+delay/2+ to suspend a goal until the input becomes rigid, and provide
modes declarations to both statically/dynamically detect problematic goals
and to (semi)automatically delay (suspend) them.
These mechanisms, however, have never been standardized.

For example, by using the \verb+delay+ pack of SWI-Prolog~\cite{SWI}, the goal
\verb+plus(X,1,Y)+ is delayed until either \verb+X+ or \verb+Y+ are
instantiated.  Delayed goals can be thought as \emph{constraints} over the
metavariables occurring in them. In the example above, the programmer is
imposing a constraint between \verb+X+ and \verb+Y+.

Going back to our diverging goal, one could imagine to provide a directive
like

\begin{verbatim}
delay (of X T) on X.
\end{verbatim}

that would delay goals of the form \verb+of X T+ whenever \verb+X+
is flexible.  As a result, instead of diverging, the running example
would terminate leaving the following (suspended) goal unsolved:

\begin{verbatim}
of x (sort I) ?- of (P x) (arr x _\ x)
\end{verbatim}

Such suspended goal is to be seen as a typing constraint on
assignments to \verb+P+: when \verb+P+ gets instantiated by a
term \verb+t+, the goal \verb+(of (t x) (arr x _\ x))+ is resumed
hence \verb+t+ is checked to be of type \verb+(arr x _\ x)+.
In turn such check can either:
\begin{itemize}
\item terminate successfully if \verb+t+ has the right type and is ground
\item fail, rejecting the proposed assignment for \verb+P+.
\item result in one or more new constraints if \verb+t+ is partial (and the non flexible part of \verb+t+ fits)
\end{itemize}
Any of these behavior matches what an elaborator does.

Extending lp with delay directive makes it possible to represent the metasenv
(as the set of delayed typing constraints) and, at the same time, avoiding the
generative semantics on predicates like \verb+of+ (that should not be as such).

\subsection{Meta theory and constraint Propagation} %%%%%%%%%%%%%%%%%%%%%%%%%%

In many situations, the constraints that accumulate over a metavariable are not independent and sets of constraints can be rewritten in order to simplify them.

For example, if \verb+P+ does not occur linearly, one could end up with
two distinct constraints it:

\begin{verbatim}
of x (sort I) ?- of (P x) (arr x _\ x)
of x (sort J) ?- of (P x) (arr (T x) y\ S x y)
\end{verbatim}

If the object language features uniqueness of typing, one surely wants to
get rid of the second constraint, and force \verb+T = a\ a+,
\verb+S = a\ b\ a+, and \verb+J = I+.
This way, when \verb+P+ gets instantiated only one goal is resumed and hence
its assigned term is checked only once.  In addition to that, when the
set of constraints is unsatisfiable one want to fail, and eventually backtrack.

In standard LP, rewriting a set of constraints is called \emph{constraint
propagation} and the programming languages that allow the user to declare
constraints and that propagate them are called \emph{Constraint Programming
Languages} (CLPs).  For example, the set $\{0 <= N <= 4, 2 <= N <= 5\}$ can be
rewritten into $\{2 <= N <= 4\}$ without changing the set of ground solutions.

 Most CLPs do not allow the user to define new constraints and propagation rules in user space. A notable exception is the first order language CHR~\cite{chr}. In CHR the user declares predicates and then gives a set of rewriting rules of the form $S_1 \setminus S_2 ~|~ G \iff S_3$ whose declarative semantics is that
 $\bigwedge S_1 \wedge G \Rightarrow (\bigwedge S_2 \iff \bigwedge S_3)$ and whose operational semantics is to match the current set of constraints against both $S_1$ and $S_2$ and, if the clause $G$ holds, replace the constraints in $S_2$ with the ones in $S_3$. All syntactic components can be omitted: $G$ defaults ot \verb+true+ and the sets $S_1,S_2,S_3$ to the emptyset. The $\iff$ symbol is also omitted when $S_3$ is, and $|$ is omitted when $G$ is. The semantics of the language is completed by a strategy that fixes the order in which propagation rules are fired and in what cases propagation rules can be backtracked. Multiple such semantics for CHR have been proposed.


Adding to lp a CHR system enables
\begin{itemize}
\item forward reasoning (i.e. propagations are thms)
\item model the optimizations given by the meta 
  theory of the object language (uniq of typing)
\item naturally relax checking into inference, for example the
the predicates \verb+<,max,succ+ with generic predicates \verb+leq,ltn+ (lax and strict inequality over an unspecified set).
\end{itemize}

Implementing in $\lambda$Prolog the part of the elaborator for ECC that deals with universe constraints is much harder and less elegant than the previous approach based on CHR. Indeed, the only solution is to keep an explicit list of constraints that become a third argument of the type-checking judgement and that needs to be threaded around the program, polluting the code and not allowing to reuse the type-checking code for the elaborator. Moreover, every time an additional constraint is added to the list, some special predicate to fire the propagation rules according to some strategy needs to be called, de facto forcing the user to implement the boilerplate code for constraint propagation.

%%%%%%%%%%%%%%%%%%%%%%%%%%%%%%%%%%%%%%%%%%%%%%%%%%%%%%%%%%%%%%%%%%%%%%%%
\section{ELPI = $\lambda$Prolog + (HO)CHR}\label{sec:elpi}

Motivated by the observations in the previous section, we implemented a new,
efficient~\cite{elpiLPAR} interpreter for $\lambda$Prolog, called \emph{ELPI}
(Embedded Lambda Prolog Interpreter) and we augmented the language with the
possibility to delay arbitrary predicates turning them into constraints, and to
propagate constraints using CHR style rewriting rules and additional
mechanisms.

In addition to the new programming constructs that deal with constraints, it
slightly differs from the version of $\lambda$Prolog implemented in Teyjus in a
few minor aspects:
\begin{itemize}
\item In ELPI all types and type and sort declarations can be omitted, whereas
	they are mandatory in Teyjus. Originally, and according
	to~\cite{jlp98}, types were necessary to implement Huet's algorithm for
	higher order unification. However, for several years now the
	unification algorithm of Teyjus only solves equations in the pattern
	fragment discovered by Miller~\cite{patternfrag}, which admits most
	general unifiers and reduce unification to a decidable problem. All
	other equations are automatically delayed by Teyjus to be fired only
	when the equation is instantiated to one in the pattern fragment. At
	the level of the implementation, the code of ELPI that implements this
	delay is shared with the one to delay arbitrary predicates.
\item The module system of Teyjus is not implemented. Only the
	\verb+accumulate+ directive is honored for backward compatibility and
	with a different semantics. In ELPI we provide instead explicit
	existentially quantified local constants whose scope spans over a set
	of program clauses. This mechanism gives in a simple way predicate and
	constructor hiding that are provided differently by the module system
	of Teyjus.
\item In a few corner cases, the parsing of an expression by Teyjus is
	influenced by the types. In particular, types are used to disambiguate
	between lists of elements and a singleton list of conjunctions. The
	syntax is disambiguated in a different way in ELPI.
\end{itemize}

Despite the differences above, we tried very hard to maintain backward
compatibility with Teyjus and its standard library. Indeed, we are able to
execute in ELPI all the code from Teyjus that we collected from the Web, up to
a very few minor changes to the source code, mostly due to 2 and 3.

\subsection{incomplete terms, delay}

The hi level directive is

\begin{verbatim}
delay (of X T) on X.
\end{verbatim}

The low level one (implemented)

\begin{verbatim}
mode (of i o).
of X T :- var X, delay (of X T) [X].
\end{verbatim}

Note that the mode makes it so that \verb+i+ arguments are never
instantiated by backchaining, so other rules (generative) would not
apply.  This is so common we provide the following syntactic sugar.

\begin{verbatim}
of (?? as X) T :- delay (of X T) [X].
\end{verbatim}

where \verb+??+ is a symbol unifying only with flexible, and
\verb+f (t as X) :- c+ de-sugars to \verb+f X :- X = t, c+,
i.e. it is a syntax to name a subterm reminiscent of what OCaml
pattern matching let one do.

Delayed goals become constraints keyed on the flexible variable
(\verb+X+ in the example above).  \emph{As soon} as \verb+X+ gets
instantiated, the delayed goal is resumed.  Symmetrically, as soon
as a contraint is added propagation is triggered.

If we go back to our previous example

\begin{verbatim}
?- of (lam a\P a) (arr (sort I) a\ arr a _\ a).
\end{verbatim}

would end with a delayed goal

\begin{verbatim}
of x (sort I) ?- of (P x) (arr x _\ x)
\end{verbatim}

a subsequent query

\begin{verbatim}
goal> P = a\ lam pa\ Q a pa .
\end{verbatim}

would resume the goal

\begin{verbatim}
of x (sort I) ?- of (lam pa\ Q x pa) (arr x _\ x)
\end{verbatim}

that would progress into the again suspended goal

\begin{verbatim}
of x (sort I), of y x ?- of (Q x y) x
\end{verbatim}

ITPs build the proof term by instantiation, and one has to check
if it fits typing.  Here one cannot forget the check, since it is
the programming language that resumes type checking whenever there is
proof progress (by uvar instantiation).

\subsection{delayed goals, HO constraint propagation}

CHR seen before does 1st order constraints, here one wants to deal
with $\lambda$Prolog goals as constraints.  Eg one wants to write  propagation
rules like this one

\begin{verbatim}
constraint of {
  rule (G ?- of X T1) \ (G ?- of X T2) <=> (G ?- conv T1 T2).
}
\end{verbatim}

it reflects uniqueness of typing: if \verb+X+ is used non linearly,
then each occurrence must have the same type.

the proof theoretic semantics of $\lambda$Prolog suggests that G has to be unified
up to commutativity of cunjunction, and that heigen variables have to
be unified as in equivariate unification (i.e. an injective map).

For example,


\begin{verbatim}
goal> of (lam a\ lam pa\ P a pa) (arr (sort I) a\ arr a _\ a),
      of (lam a\ lam pa\ P a pa) (arr (sort J) a\ arr a _\ a).
\end{verbatim}

generates 2 suspended goals

\begin{verbatim}
of x (sort I), of y x ?- of (P x y) x
of w (sort J), of z w ?- of (P w z) w
\end{verbatim}

........

of course one would like G to be compared like that

\begin{verbatim}
constraint of {
  map _ [] [].
  map F [X|XS] [Y|YS] :- F X Y, map F XS YS.
  conv-of (of X T1) (of X T2) :- conv T1 T2.

  rule (nabla xv\ G2 xv ?- of (X xv) T1)
     \ (nabla yv\ G2 yv ?- of (X yv) T2)
     > (xv ~ yv)
   % after that point, G1 G2 T1 T2 are aligned on the names
   % and G1 G2 are sorted accordingly (or map is made more complex)
     | (pi zv\ map conv-of (G1 zv) (G2 zv))
     <=> (nabla zv\ G1 zv ?- conv T1 T2).
}
\end{verbatim}

or even better to have such semantics built-in in the former example
but what if outside llam?


DIRE CHE BASTA AGGIUNGERE un \$delay MA FORSE QUI CI VORREBBE QUELLO AD ALTISSIMO LIVELLO CHE NON ABBIAMO

FARLO VEDERE SULL'ESEMPIO?

AGGIUNGERE ESEMPIO SU PROPAGAZIONE CONSTRAINTS A LA CHR?

%%%%%%%%%%%%%%%%%%%%%%%%%%%%%%%%%%%%%%%%%%%%%%%%%%%%%%%%%%%%%%%%%%%%%%%%
\section{Application}

\subsection{Kernel for CIC}\label{sec:kernel}

The implementation of weak head reduction given before is well-known to be very inefficient, because the body of the abstraction of a $\beta$-redex is immediately and completely traversed when the redex is fired, unless the reduction of $\lambda$Prolog is implemented via explicit substitutions like in Teyjus. Moreover, the strategy implemented by \verb+whd*+ is call-by-name, that is inefficient per se.

To speed up reduction, we now provide a different implementation based on the reduction machine for call-by-need called Wadsworth Abstract Machine~\cite{beniamino}. The machine is a modified Krivine Abstract Machine (KAM~\cite{kam}) that records in the machine environment both terms and their normal forms, the latter being computed lazily by a new invocation of the reduction machine that is fired only when the normal form is required for the first time, i.e. when the variable bound to that value occurs in head position during reduction.

In standard implementations, variables are represented by De Bruijn indexes and machine environments are stacks, indexed by the variables. Since we do not use and do not want to use indexes, we take a different approach. The machine environment is represented by a set of \verb+(val N T V NF)+ assumptions that we dynamically add to the $\lambda$Prolog environment using logical implication. Their meaning is that to the variable \verb+N+ we associate a value \verb+V+ of type \verb+T+ and its normal form \verb+NF+. The latter is initialised with a fresh metavariable when the binding for \verb+N+ is destroyed by a $\beta$-reduction, and it is filled in when the value is requested for the first time. To compute the term to be aligned to \verb+NF+ we first start a new machine on \verb+V+ and then we decode the final machine state, an operation that we call \emph{unwind}.  We call a variable \verb+N+ for which a \verb+val N _ _ _+ assumption is present \emph{val-bound}.

Fetching the normal form of \verb+x+ from the environment via a call to \verb+val x _ _ NF+ amounts to fetching a clause from the current program, that is performed efficiently in any reasonable implementation of $\lambda$Prolog thanks to clause indexing.  ELPI, as most Prolog engines, indexes clauses on the head predicate \emph{and its first argument}: since \verb+x+ is a fresh constant there is little risk of having collisions.  Even if the indexing ignores the predicate argument the cost of the lookup is at worst $O(n)$ where $n$ is the size of the reduction machine environment.  In line with what one would obtain by representing the environment as a linked list and variables with De Bruijn indexes.

An assumption $A$ introduced in $\lambda$Prolog via $A \Rightarrow B$ is only visible in $B$. Therefore, once a variable is bound via an assumption $A$ of the form \verb+(val x t n nf)+, we need either 1) to be sure that the rest of the computations that requires \verb+x+ is performed in $B$; or 2) to reintroduce syntactically the binding again around every term that escapes the scope of the assumption.

We force 1) by coding the reduction machine in Continuation Passing Style (CPS): the \verb+whd1+ predicate takes in input the head and stack of the machine together with a continuation \verb+K+, and apply \verb+K+ to the new machine head and stack. Moreover, it also passes to \verb+K+ the list of val-bound variables.
The \verb+whd*+ predicate is also implemented in CPS style, and it is responsible for composing together the lists of val-bound variables.

Such list of variables is used when reduction is over to end the CPS style by performing 1).  For this purpose we introduce the term constructor for local abbreviations \verb+abbr+ (also called \verb+let .. in+ in several functional programming languages).  For example the \verb+whd_unwind t NF+ predicate computes the normal form \verb+NF+ of \verb+t+ by calling \verb+whd*+ on \verb+t+ and passing a continuation that unwinds the final machine state by introducing the explicit binder \verb+abbr ty val n \ ..+ for each val-bound variable \verb+n+.  The signature and type-checking rule for \verb+abbr+ follow:

\begin{Verbatim}
type abbr term -> term -> (term -> term) -> term.  % local definition (let-in)

of (abbr TY TE F) (abbr TY TE G) :-
 of TE TY', sub TY' TY, pi x \ of x TY => val x TY TE NF => of (F x) (G x).
\end{Verbatim}

The code of the reduction machine is the following:

\begin{Verbatim}
type whd1 term -> list term -> (list var -> term -> list term -> prop) -> prop.

% KAM-like rules in CPS style
whd1 (app M N) S K :- K [] M [N|S].
whd1 (lam T F1) [N|NS] K :- pi x \ val x T N NF => K [x] (F1 x) NS.
whd1 X S K :- val X _ N NF, if (var NF) (whd_unwind N NF), K [] NF S.

% reflexive, transitive closure
whd* T1 S1 K :- whd1 T1 S1
 (vl1 \ t1 \ s1 \ whd* t1 s1
  (vl2 \ t2 \ s2 \ sigma VL \ append vl1 vl2 VL, K VL t2 s2)), !.
whd* T1 S1 K :- K [] T1 S1.

% Whd followed by machine unwinding.
type whd_unwind term -> term -> prop.
whd_unwind N NF :-
 whd* N [] (l \ t \ s \ sigma TS \ unwind_stack s t TS, put_abbr l TS NF).

% unwind_stack takes an head and a stack and decodes them to a term
unwind_stack [] T T.
unwind_stack [X|XS] T O :- unwind_stack XS (app T X) O.

% put_abbr takes a list of variables and a term and wraps the latter
% with local definitions for the variables in the list
put_abbr [] NF NF.
put_abbr [X|XS] I (abbr T N K) :- val X T N _, put_abbr XS I (K X).
\end{Verbatim}

The predicate \verb+match_sort+ is implemented trivially. The one for \verb+match_arr+ is slightly more delicate: the input \verb+T+ is put in weak head normal form by \verb+whd*+; the continuation retrieves the term and checks that it is a dependent product, projecting out the two subterms; the subterms may contain variables bound in the environment; therefore, before returning them, it is necessary to close them using technique 2) above by means of \verb+abbr+s via \verb+put_abbr+, like \verb+whd_unwind+ does.

\begin{Verbatim}
type match_sort term -> @univ -> prop.
match_sort T I :- whd* T [] (l \ t \ s \ t = sort I, s = []).

type match_arr term -> term -> (term -> term) -> prop.
match_arr T A F :-
 whd* T [] (l \ t \ s \ sigma A' \sigma F' \
  s = [], t = arr A' F', put_abbr l A' A,
  pi x \ put_abbr l (F' x) (F x)).
\end{Verbatim}

Efficient reduction is not sufficient to obtain a quick term comparison
routine.  In particular both Matita and Coq implement the same heuristic: in
place of putting the two terms to be compared in weak head normal form
immediately, the two terms are lazily reduced on demand during comparison, that
at every step tries $\alpha$-conversion (i.e. $\lambda$Prolog equality) and
congruence rules to avoid unnecessary reductions.

In order to perform weak head reductions lazily, we obtain \verb+conv+ and \verb+sub+ as two instances of a \verb+comp+ comparison predicate that works on two reduction machine statuses. The third argument is \verb+eq+ if the intended comparison is \verb+conv+ while \verb+leq+ for \verb+sub+.
\begin{Verbatim}
type conv term -> term -> prop.
type sub term -> term -> prop.
type comp term -> list term -> eq_or_leq ->  term -> list term -> prop.

conv T1 T2 :- comp T1 [] eq T2 [].
sub T1 T2 :- comp T1 [] leq T2 [].

% fast path + axiom rules
comp T1 S1 _ T1 S1 :- !.
comp (sort I) [] leq (sort J) [] :- lt I J.
% congruence + fast path rule
comp X S1 _ X S2 :- map S1 S2 conv, !.
% congruence rules
comp (lam T1 F1) [] _ (lam T2 F2) [] :- conv T1 T2, pi x \ conv (F1 x) (F2 x).
comp (arr T1 F1) [] D (arr T2 F2) [] :- conv T1 T2, pi x \ comp (F1 x) [] D (F2 x) [].
% reduction rules
comp T1 S1 D T2 S2 :- whd1 T1 S1 (_ \ t1 \ s1 \ comp t1 s1 D T2 S2), !.
comp T1 S1 D T2 S2 :- whd1 T2 S2 (_ \ t2 \ s2 \ comp T1 S1 D t2 s2), !.
\end{Verbatim}

\paragraph{Extensions with global definitions, declarations, primitive inductive types, case analysis and structural recursive definitions.}
We implemented all the extensions mentioned in a modular way. To activate one extension, it is sufficient to accumulate in the kernel a $\lambda$Prolog file that adds new constructors to the \verb+term+ and to the reduction machine stack type, and the relative clauses to the \verb+whd1+, \verb+conv+ and \verb+of+ predicates.

We omit the implementation of the extensions in the paper. All together, they provide a kernel that is functionally equivalent to the one of Matita. The reduction strategies implemented differ though, the one of Matita being more complex than call-by-need. We plan in the future to synchronise the two strategies in order to compare the performance of the two kernels.

To test the kernel, we branched it to Matita, feeding it with every term that is type-checked by Matita when checking the arithmetic library of the system.
According to our measurements the kernel presented in the paper is 15 times slower than the interpreted version of Matita (to be fair versus ELPI that is an interpreter). The trade off is that the code is more elegant, much simpler and shorter (e.g. 91 lines saved just from the lifting of De Bruijn indexes). A previous implementation of a type-checker for Automath was only 5 times slower~\cite{elpiLPAR}, suggesting the possibility to optimize the code further.


\subsection{Prototype elaborator}\label{sec:elaborator}
We are developing an elaborator for CIC as a modular extension of the kernel described in Sect.~\ref{sec:kernel}. The elaborator mimics as close as possible the behaviour of the one of Matita 0.99.1~\ref{xxx}, with the exception of the handling of universe constraints that follows Coq XXX~\ref{yyy}.

An alternative very promising choice would have been to mimic the elaborator described in~\ref{zzz} and already presented via typing rules that yield a set of higher order unification constraints to be later solved. In the future, the choice to be closer to Matita will allow us to easily compare the performances of the elaborator written in ELPI with the one of Matita written in OCaml, in order to further optimize the ELPI interpreter.

To implement the elaborator, the following steps need to be taken at the beginning of a file that accumulates at the end the code of the kernel and of the file implementing the PTS rules.
\begin{itemize}
\item The accumulated code for the PTS is not the ones accumulated by the kernel that just verifies the typing inequations, but the one based on the first order CHR rules presented in Section~\ref{chruniverses}.
\item The following predicates defined by structural recursion over a term
 must be given a suitable mode to prevent the generative behaviour:
 AAA (type checking), BBB (reduction), CCC (conversion), plus all derived
 predicates like DDD that infers a type and checks that the inferred type is
 a sort.
\item A line is inserted to delay type checking a flexible term. A suitable set of propagation rules in CHR style can be added to capture unicity of typing.
\item Conversion when at least one of the arguments is a flexible term amounts to higher order narrowing. Instead of delaying the goal, an heuristic already used in Matita immediately rewrites the constraint by solving the unification problem. The heuristic, in the case $M~t = s$ tries to instantiates $M$ with
$\lambda x. s[x/t]$ where the latter operation replaces with $x$ subterms of $s$ that unify with $t$. In practice, in Huet's terminology it always prefer projection to mimic. When $M$ is applied to multiple $t$s or to flexible $t$s, further heuristics are used to pick one of the solution. These heuristics are clearly incomplete, discarding all solutions but one and sometimes failing to find a solution when it exists. However, they are more general than the ones implemented in Coq and, from a practical perspective, they guess most of the time the unifier that is expected by the user when interactive with the ITP.
\item In place of delaying reducing a flexible term, the term is returned in the current state, weakening the semantics of the the BBB predicate. In practice, this is not a problem because the result of reduction is then always passed to the conversion predicate.
\end{itemize}

\begin{verbatim}
mode (t+step i o).
mode (has+sort i o).
mode (conv+whnf i i i i i).
t+step (?? as K) T :- !, $constraint (t+step K T) K.
has+sort (?? as S) T :- !, $constraint (has+sort S T) S.

conv+whnf A B C D E :- $print "##" (conv+whnf A B C D E), fail.

% Bug1: ci possono essere let-in sulle var in Ti
% Bug2: potrebbe servire un <= se il secondo è una sorta
%conv+whnf (?? as T1) [] _M T2 L2 :- !, $llam_unif T1 {zip T2 L2}.
conv+whnf T1 L1 _M (?? as V) L2 :- !, bind-list L2 {zip T1 L1} V.

conv+whnf (?? as T1) L1 M T2 L2 :- !, $constraint (conv+whnf T1 L1 M T2 L2) T1.
conv+whnf T1 L1 M (?? as T2) L2 :- !, $constraint (conv+whnf T1 L1 M T2 L2) T2.

constraint t+step has+sort conv+whnf r+step {
}

%%% library
mode (zip i i o).
zip HD [] HD :- !.
zip (appl HD TL) Args (appl HD TLArgs) :- !, append TL Args TLArgs.
zip HD Args (appl HD Args).

is_rigid C :- $is_name C. % ; C = const _ ; C = indt _ ; C = indc _

mode (bind-list i i o).
bind-list [] T T' :- copy T T'.
bind-list [ ?? |VS] T R :- !, pi x\ bind-list VS T (R x).
bind-list [appl C AS | VS] T R :- is_rigid C, !,
  pi x\ (pi L\ copy (app[C|L]) x :- conv+args L AS) => bind-list VS T (R x).
bind-list [C|VS] T R :- is_rigid C, !,
  pi x\ copy C x => bind-list VS T (R x).

mode (copy i o).
copy A B :- $print "AA" (copy A B), fail.
copy X Y :- r+step X Y _. % Bang or not???
copy X Y :- $is_name X, X = Y, !.
copy X Y :- $is_name X, r+step X T _, !, copy T Y.
copy (sort _ as C) C :- !.
copy (appl X1 L1) (appl X2 L2) :- copy X1 X2, $print "UU", map copy L1 L2.
\end{verbatim}

\subsection{universes}


For example, the rules for type-checking ECC call the predicates \verb+<,max,succ+ over integers. The corresponding elaborator may be faced with sorts like \verb+sort M+ where \verb+M+ is a metavariable. The elaborator is asked to compute a suitable assignment to the metavariables that respects all constraints. Without breaking consistency, the successor predicate can be relaxed to a strictly less than, the max to a generic upper bound and the set of integers to an unspecified partial order, de facto replacing the predicates \verb+<,max,succ+ with generic predicates \verb+leq,ltn+ (lax and strict inequality over an unspecified set). Finally, the two predicates, once turned into constraints, admit the following complete set of CHR propagation rules able to detect strict cicles that correspond to unsatisfiability of the constraints.

\label{chruniverses}
\begin{verbatim}
succ I J :- ltn I J.
I < J :- ltn I J.
max I J K :- leq I K, leq J K.

constraint leq ltn {
  % incompatibility
  rule (leq X Y) (ltn Y X) <=> false.
  rule (ltn X Y) (ltn Y X) <=> false.
  rule (ltn X X) <=> false.
  
  % reflexivity
  rule \ (leq X X).

  % antisymmetry
  rule (leq X Y) \ (leq Y X) <=> (Y = X).

  % transitivity
  rule (leq X Y) (leq Y Z) <=> (leq X Z).
  rule (leq X Y) (ltn Y Z) <=> (ltn X Z).
  rule (ltn X Y) (leq Y Z) <=> (ltn X Z).
  rule (ltn X Y) (ltn Y Z) <=> (ltn X Z).

  % idempotence
  rule (leq X Y) \ (leq X Y).
  rule (ltn X Y) \ (ltn X Y).
}
\end{verbatim}


%%%%%%%%%%%%%%%%%%%%%%%%%%%%%%%%%%%%%%%%%%%%%%%%%%%%%%%%%%%%%%%%%%%%%%%%
\section{Related works}
\begin{enumerate}
\item Costrutti tipo delay
\item Logical frameworks (Dedukti, MMT, etc.) e brrr Isabelle
\item Da qualche parte: il pre-print by Abel sull'elaborator basato su
 constraint e lean che implementa un constraint solver di qualche tipo
\end{enumerate}

\section{SCALETTA}

Prima possibilità:
\begin{enumerate}
\item Si introduce per bene (??) la sintassi/semantica di CHR
\item Si fa notare quanto sia costoso il passo di firing di una regola CHR
      e si conclude che è meglio identificare sotto-frammenti
\item Si introducono i mode + il costrutto delay + i pattern sulle metavariabili
      spiegandoli come il caso comune nel quale un constraint sia immediatamente
      propagabile senza doverne cercare altri
\item (??) Si aggiungono gli altri costrutti di ELPI
\item Kernel modulare di Ferruccio ed esperimenti con Matita
\item Estensione con universi flottanti e refiner
\end{enumerate}

Seconda possibilità: si parte dal kernel di Ferruccio e poi ``on-demand'' si introduce il refiner descrivendo i comandi che usiamo 

\label{sect:bib}
\bibliographystyle{plain}
\bibliography{bib}

\end{document}
